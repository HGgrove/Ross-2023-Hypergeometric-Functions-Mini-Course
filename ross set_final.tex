\documentclass[12pt]{amsart}

\usepackage{enumitem,color, amssymb}
\usepackage{tikz}
\usepackage{verbatim}
\usepackage{hyperref}
\usepackage{enumerate}
\usepackage{geometry}
\usepackage{soul}
\geometry{legalpaper, portrait, margin=.5in}


\newtheorem{theorem}{Theorem}[section]
\newtheorem{lemma}[theorem]{Lemma}
\newtheorem{corollary}[theorem]{Corollary}
\newtheorem{proposition}[theorem]{Proposition}
\newtheorem{noname}[theorem]{}
\newtheorem{sublemma}{}[theorem]
\newtheorem{conjecture}[theorem]{Conjecture}


\theoremstyle{definition}
\newtheorem{definition}[theorem]{Definition}
\newtheorem{example}[theorem]{Example}

\theoremstyle{remark}
\newtheorem{remark}[theorem]{Remark}

\numberwithin{equation}{section}
\newcommand{\notdivides}{\nmid}
\newcommand{\ba}{\backslash}
\newcommand{\utf}{Unicode Transformation Format}
\newcommand{\del}{\setminus}
\newcommand{\con}{\,/\,}
\def \l {\lambda}
\newcommand{\bl} {\color{blue}}
\newcommand{\bk} {\color{black}}

\def\C{\mathbb C}
\def\G{\mathbf G}
\def\A{\mathbf A}
\def\K{\mathbf K}
\def\bk{\mathbf k}

\def \Ind {\textnormal{Ind}}

\def\g{\gamma}
\def \gal{\textnormal{Gal}}
\def\GL{\mathbf{GL}}

\def\Tr{{\rm Tr}}
\def\Gal{{\rm Gal}}
\def\Frob{{\rm Fr}}
\def\F{{\mathbb F}}
\def\Fr{{\F_{r}}}  \def\Fq{{\F_{q}}}
\def\Z{{\mathbb Z}}

\def\Q{{\mathbb Q}}


\def\O{\mathcal O}


\def \R {\mathcal R}
\def\G{\Gamma}

\def\ord{{\mathrm ord}}

\def\FH{\mathfrak H}
\def\Im{\mathrm{Im}\,}
\def\SL{\mathrm{SL}}
\def\J{\mathrm{Jac}}
\def\M#1#2#3#4{\begin{pmatrix}#1&#2\\#3&#4\end{pmatrix}}
\def\SM#1#2#3#4{\left(\begin{smallmatrix}#1&#2\\#3&#4\end{smallmatrix}\right)}
\def\N{\mathbb N}

\def\({\left(}
\def\){\right)}
\def \l {\lambda}

\def\p{\varphi}




\def\CC#1#2{\binom {#1}{#2}}
\def\C{\mathbb{C}}
\def\R{\mathbb{R}}
\def\Z{\mathbb{Z}}
\def\Q{\mathbb{Q}}
\def\F{\mathbb{F}}
\def\O{\mathcal{O}}
\def\H{\mathbb H}
\def\M#1#2#3#4{\begin{pmatrix}#1&#2\\#3&#4\end{pmatrix}}
\def\tr{\mathit{tr}}
\def \Frob{\text{Frob}}
\def \ol{\overline}
\def \eps{\varepsilon}
\newcommand{\fp}
{\mathbb{F}_p}

\def \g {\mathfrak{g}}

\newcommand{\fq}
{\mathbb{F}_q}

\newcommand{\fpc}
{\mathbb{F}_p^{\times}}

\newcommand{\fphat}
{\widehat{\mathbb{F}_{p}^{\times}}}

\newcommand{\fqhat}
{\widehat{\mathbb{F}_{q}^{\times}}}



\newcommand*\HYPERskip{&}
\catcode`,\active
\newcommand*\pFq{
\begingroup
\catcode`\,\active
\def ,{\HYPERskip}%
\doHyper
}
\catcode`\,12
\def\doHyper#1#2#3#4#5{%
\, _{#1}F_{#2}\left[\begin{matrix}#3 \smallskip \\  #4\end{matrix} \; ; \; #5\right]%
\endgroup
}


\def\gen#1{\langle #1\rangle}


\newcommand*\HYPERp{&}
\catcode`,\active
\newcommand*\pPq{
	\begingroup
	\catcode`\,\active
	\def ,{\HYPERp}%
	\doHyperP
}
\catcode`\,12
\def\doHyperP#1#2#3#4#5{%
	\, _{#1}{P}_{#2}\left[\begin{matrix}#3 \smallskip \\  #4\end{matrix} \; ; \; #5\right]%
	\endgroup
}



\newcommand{\hg}[4]{
{}_{2}F_{1} \left[
\begin{matrix}
#1 & #2 \smallskip \\
   & #3 \\
\end{matrix}
 \; ; \; #4
\right]
}



\newcommand{\hgq}[4]{
\,_{2} \mathbb F_{1} \left(
\begin{matrix}
#1 & #2 \\
   & #3 \\
\end{matrix}
 \; ; \; #4
\right)_q
}

\newcommand{\hgp}[4]{
_{2}\mathbb F_{1} \left(
\begin{matrix}
#1 & #2 \\
   & #3 \\
\end{matrix}
 \; ; \; #4
\right)_p
}



\newcommand{\phg}[4]{
{}_{2}\mathbb P_{1} \left[
\begin{matrix}
#1 & #2 \\
   & #3 \\
\end{matrix}
 \; ; \; #4
\right]
}





\newcommand*\HYPER{&}
\catcode`,\active
\newcommand*\pFFq{
\begingroup
\catcode`\,\active
\def ,{\HYPER}%
\doHyperF
}
\catcode`\,12
\def\doHyperF#1#2#3#4#5{%
\, _{#1}{\mathbb F}_{#2}\left[\begin{matrix}#3 \smallskip \\  #4\end{matrix} \; ; \; #5\right]%
\endgroup
}


\newcommand*\HYPERpp{&}
\catcode`,\active
\newcommand*\pPPq{
\begingroup
\catcode`\,\active
\def ,{\HYPERpp}%
\doHyperFpp
}
\catcode`\,12
\def\doHyperFpp#1#2#3#4#5{%
\, _{#1}{\mathbb P}_{#2}\left[\begin{matrix}#3 \smallskip \\  #4\end{matrix} \; ; \; #5\right]%
\endgroup
}



\newcommand{\hgthree}[6]{
\,_{3}\mathcal P_{2} \left(
\begin{matrix}
#1 & #2 & #3 \\
   & #4 & #5 \\
\end{matrix}
 \; ; \; #6
\right)
}

\begin{document}


\title{Hypergeometric Functions Course \\ Ross 2023 \\ Problem Set}

\author{Brian Grove}



\maketitle

\begin{center}
\large\textbf{\underline{Week I:}}
\end{center}

\vspace{8mm}

\section*{Lecture 1 Problems}



\begin{itemize}

    
    
    
    \item[\textbf{1.}] Recall that 

    $$\pFq{2}{1}{\frac{1}{2}&\frac{1}{2}}{&1}{z}:= \sum_{k=0}^{\infty} \frac{\big(\frac{1}{2})_{k}\big(\frac{1}{2})_{k}}{(1)_{k}(1)_{k}}z^{k}$$

    for $z \in \C$ with $|z| < 1$.
    \\

    \begin{itemize}
        \item[(a)] Show $$\pFq{2}{1}{\frac{1}{2}&\frac{1}{2}}{&1}{z} = \sum_{k=0}^{\infty} \binom{2k}{k}^{2} \bigg(\frac{z}{16} \bigg)^{k}.$$

        \vspace{8mm}

        \textbf{Note: In parts $(b)-(d)$ prove, disprove, or salvage if possible.}

        \vspace{8mm}

        \item[(b)] Use part (a) to show that 

        $$\pFq{2}{1}{\frac{1}{2}&\frac{1}{2}}{&1}{4} = \sum_{k=0}^{\infty} \frac{1}{(k!)^2}.$$

        \vspace{4mm}

        \item[(c)] Find lower and upper bounds for $$\pFq{2}{1}{\frac{1}{2}&\frac{1}{2}}{&1}{4}$$ using part (b).

        \vspace{4mm}

        \item[(d)] Simplify $$\pFq{2}{1}{\frac{1}{2}&\frac{1}{2}}{&1}{x}$$ for $x=\frac{1}{4}, 32, \frac{1}{16},$ and $ \frac{4}{3}$ to the best of your ability. Do you notice any patterns? What lower and upper bounds can you prove in these additional cases?
    \end{itemize}

    \vspace{2mm}

    \hrule{}
    
    \vspace{8mm}

     
    \item[\textbf{2.}] Define $$\pFq{0}{1}{*}{a}{z}:= \sum_{k=0}^{\infty} \frac{z^k}{(a)_{k} \cdot k!}$$

    for $z \in \C$ such that $|z| < 1$.
    \\

    Recall the Maclaurian series for cosine and sine,

    $$\cos(x) = \sum_{k=0}^{\infty} (-1)^{k} \frac{x^{2k}}{(2k)!}$$

    and

    $$\sin(x) = \sum_{k=0}^{\infty} (-1)^{k} \frac{x^{2k+1}}{(2k+1)!}.$$
    \vspace{3mm}
    
    \begin{itemize}
    \item[(a)] Show that 

    $$\cos(x) = \pFq{0}{1}{*}{\frac{1}{2}}{-\frac{x^{2}}{4}}$$

    and

    $$\sin(x) = x \cdot \pFq{0}{1}{*}{\frac{3}{2}}{-\frac{x^{2}}{4}}$$

    for $x \in \R$ such that $|x| < 1$.
    \vspace{4mm}

    \item[(b)] How are the proofs for $\cos(x)$ and $\sin(x)$ related? Does one of the formulas imply the other or are the arguments distinct?

    \vspace{4mm}

    \item[(c)] Recall the double angle and Pythagorean identities from trigonometry, $\sin(2 \theta) = 2 \sin(\theta) \cos(\theta)$ and $\sin^{2}(\theta) + \cos^{2}(\theta) = 1$, respectively.
    \\

    Rewrite these two identities for sine and cosine in terms of the $_{0}F_{1}$ function with part (a). Then use the $_{0}F_{1}$ version of the double angle formula to compute $\sin(\frac{\pi}{3})$ in terms of $_{0}F_{1}$ functions.
    \end{itemize}

    \vspace{2mm}

    \hrule{}

    \vspace{8mm}

    \item[\textbf{3.}] The power series representations for $\sin^{-1}(x)$ and $-\ln(1-z)$ are

    $$\sin^{-1}(x) = \sum_{k=0}^{\infty} \frac{(2k)!}{2^{2k}(k!)^{2}} \frac{x^{2k+1}}{2k+1}$$

    for $x \in \R$ such that $|x| < 1$ and 

    $$-\ln(1 - z) = \sum_{k=0}^{\infty} \frac{z^{k+1}}{k+1}$$

    for $z \in \C$ such that $|z| < 1$.

    \vspace{4mm}

    \begin{itemize}
     \item[(a)] Show that 

     $$\sin^{-1}(x) = x \cdot \pFq{2}{1}{\frac{1}{2}&\frac{1}{2}}{&\frac{3}{2}}{x^2}$$

     when $|x| < 1$.

     \vspace{4mm}

     \item[(b)] Substitute $x = \frac{1}{\sqrt{2}}$ into the equation in part $(a)$ to express $\pi$ as a $_{2}F_{1}$ value.

     \vspace{4mm}

     \item[(c)] Compare your answer from part (b) to the formula for $\pi$ we proved in the lecture,

     $$\pi = 4 \cdot \pFq{2}{1}{\frac{1}{2}&1}{&\frac{3}{2}}{-1},$$

     to conclude that 

     $$\pFq{2}{1}{\frac{1}{2}&1}{&\frac{3}{2}}{-1} = \frac{1}{\sqrt{2}} \cdot \pFq{2}{1}{\frac{1}{2}&\frac{1}{2}}{&\frac{3}{2}}{\frac{1}{2}}.$$

     \vspace{4mm}

     \item[(d)] Show that 

     $$-\ln(1-z) = z \cdot \pFq{2}{1}{1&1}{&2}{z}$$

     for $|z| < 1$.

     \vspace{6mm}

     \item[(e)] Substitute $z = 1 - \frac{e}{4}$ into the result of part (d) to write $\frac{1}{e-4}$ as a $_{2}F_{1}$ value.
    \end{itemize}

    \vspace{2mm}

    \hrule{}

    \vspace{8mm}

    \item[\textbf{4.}] Is the operation $*$, where $P*Q$ means the third point on the line through the points $P$ and $Q$ on an elliptic curve, a group operation? If so, describe the identity and inverse elements.

    \vspace{5mm}

    \hrule{}

    \vspace{8mm}

    \item[\textbf{5.}] Prove the addition of two points on an elliptic curve is closed under the elliptic curve addition law.

    \vspace{2mm}

    \hrule{}

    \vspace{8mm}

    \item[\textbf{6.}] Prove the associativity of addition for the elliptic curve addition law with a picture.

    \vspace{2mm}

    \hrule{}



    
    
\end{itemize}

\section*{Lecture 2 Problems}


\begin{itemize}

\item[\textbf{1.}] Let $s \in \C$. The gamma function is defined as 

    $$\Gamma(s):= \int_{0}^{\infty} t^{s-1}e^{-t} \, dt$$ when $\text{Re}(s)>0$.
    
    \vspace{4mm}

    \begin{itemize}
        \item[(a)] Show that $\Gamma(1) = 1$ and $\Gamma(s+1) = s \cdot \Gamma(s)$ for $s \in \C \setminus \{\Z_{\leq 0}\}$ and $\text{Re}(s)>0$.

        \vspace{4mm}

        \item[(b)] Use part $(a)$ to show that $\Gamma(k) = (k-1)!$, where $k$ is a positive integer.

        \vspace{4mm}

        \item[(c)] Another common way to define the gamma function is as

        $$\Gamma(s) = \lim_{s \to \infty} \frac{k^{s-1}k!}{(s)_{k}} $$

        for all $s \in \C \setminus \{\Z_{\leq 0}\}.$
        \\

        Use the above product definition of the gamma function to give another proof of the results in parts $(a)$ and $(b)$. 

        \vspace{4mm}

        \item[(d)] Why is the restriction that $s \notin \Z_{\leq 0}$ needed in the limit definition of $\Gamma(s)$? Try some examples, if necessary.

        \vspace{4mm}

        \item[(e)] Do you think it is easier to prove parts $(a)$ and $(b)$ with the integral or limit definition of $\Gamma(s)$? How are the two proofs similar and how do they differ?

        \vspace{4mm}
        
        \hrule{}
    \end{itemize}

    \vspace{4mm}

\item[\textbf{2.}] The reflection formula for the gamma function states that 

$$\Gamma(s)\Gamma(1-s) = \frac{\pi}{\sin(\pi s)}$$

for $s \in \C \setminus \Z$.
    \vspace{4mm}

    \item[(a)] Why is the reflection formula \textbf{not} valid for integers? Try some examples, if necessary.

    \vspace{4mm}

    \item[(b)] Use the reflection formula to show $\Gamma(\frac{1}{2})^{2} = \pi$. 

    \vspace{4mm}

    \item[(c)] Show the gamma function is always positive when $\text{Re}(s) > 0$.

    \vspace{4mm}

    \item[(d)] Conclude that $\Gamma \big(\frac{1}{2}\big) = \sqrt{\pi}$ from parts (b) and (c).

    \vspace{4mm}

    \item[(e)] Use the reflection formula to compute $\Gamma(\frac{n}{2})$ for $n=3,5,7$. Guess a general formula for $\Gamma(\frac{2n+1}{2})$ when $n \geq 1$. Can you prove your general formula?

    \vspace{4mm}

    \hrule{}

    \vspace{8mm}

\item[\textbf{3.}] Show that $\Gamma(s)$ has no zeros. [\textit{Hint: The reflection formula may be useful.}]

\vspace{4mm}

\hrule{}

\vspace{8mm}

\item[\textbf{4.}] Define the Beta function as 

$$B(x,y) = \int_{0}^{1} t^{x-1}(1-t)^{y-1} \, dt$$ for $\text{Re}(x), \text{Re}(y) > 0$.

\vspace{6mm}

Show that $$B(x,y) = \frac{\Gamma(x)\Gamma(y)}{\Gamma(x+y)}$$ when $\text{Re}(x), \text{Re}(y) > 0$.

\vspace{4mm}

\hrule{}

\vspace{8mm}

\item[\textbf{5.}] \textbf{In this problem prove, disprove, and salvage if possible for each statement}.

\vspace{4mm}

\begin{itemize}
\item[(a)] $\Gamma(0)$ is undefined.
\vspace{3mm}

\item[(b)] $B(-\frac{1}{2}, -\frac{1}{2}) = \frac{\Gamma(-\frac{1}{2})\Gamma(-\frac{1}{2})}{\Gamma(-1)} = 4 \pi$.
\vspace{3mm}

\item[(c)] $\pFq{2}{1}{\frac{1}{3}&4}{&\frac{3}{2}}{1}$ converges with the sum definition of the $_{2}F_{1}$ function.
\vspace{3mm}

\item[(d)] 

$$B \bigg(\frac{1}{3}, \frac{2}{3}\bigg) = \frac{2\pi}{\sqrt{3}}.$$
\vspace{3mm}


\end{itemize}

\vspace{4mm}

\hrule{}

\vspace{8mm}

\item[\textbf{6.}] Let $a \in \Q$ and recall the binomial series $(1-z)^{-a}$, where $z \in \C$. Show that 

$$(1-z)^{-a} = \sum_{k=0}^{\infty} \frac{(a)_{k}}{k!} z^{k}$$

when $|z| < 1$, the Maclaurian series.


\vspace{4mm}

\hrule{}

\vspace{8mm}

\item[\textbf{7.}] In the lecture, we showed that the real period of a Legendre elliptic curve is related to a $_{2}F_{1}$ value. This elliptic integral is a special case of more a general phenomenon called periods, introduced by Kontsevich and Zagier. Periods show up in many parts of number theory, algebraic geometry, topology, and beyond. Read \underline{\href{https://www.maths.ed.ac.uk/~v1ranick/papers/kontzagi.pdf}{Periods}} if you want to learn more about periods.

\vspace{4mm}

\hrule{}

\vspace{8mm}

\item[\textbf{8.}] Define the Fibonacci sequence as $F_{0} = 0$, $F_{1} = 1$, with $F_{n+2} = F_{n} + F_{n+1}$ for all integers $n$. Similarly, define the Lucas sequence as $L_{0} = 2$, $L_{1} = 1$, and $L_{n+1} = L_{n} + L_{n-1}$ for all integers $n$. All statements below are valid for all integers $n$.

\vspace{5mm}

\begin{itemize}
    \item[(a)] Show that $F_{-n} = (-1)^{n-1}F_{n}$ and $L_{-n} = (-1)^{n} L_{n}$.

    \vspace{3mm}

    \item[(b)] Show that $L_{n} = F_{n+1} + F_{n-1}$ and $F_{n} = \frac{1}{5}(L_{n+1} + L_{n-1})$.

    \vspace{3mm}

    \item[(c)] In the lecture we proved the Binet formula for $F_{n}$. Prove the Binet formula for $L_{n}$,

    $$L_{n} = \bigg(\frac{1+\sqrt{5}}{2} \bigg)^{n} + \bigg(\frac{1-\sqrt{5}}{2} \bigg)^{n}.$$

    \vspace{3mm}

    \item[(d)] Let $p$ be an odd prime. Show that 

    $$L_{p} \equiv 1 \mod p$$

    and

    $$F_{p} \equiv \bigg(\frac{5}{p} \bigg) \mod p,$$

    \vspace{4mm}

    where $\big(\frac{a}{p} \big)$ denotes the Legendre symbol.
\end{itemize}

\vspace{4mm}

\hrule{}


\end{itemize}

\section*{Lecture 3 Problems}

\begin{itemize}

\item[1.] Use the Pfaff transformation for the $_{2}F_{1}$ function to prove the Euler transformation for the $_{2}F_{1}$ function.

\vspace{4mm}

\hrule{}

\vspace{8mm}

\item[2.] Recall the Gauss evaluation formula for the $_{2}F_{1}$ function at $z=1$ from the lecture.
\\

\textbf{Use this formula to prove, disprove, and salvage if possible for the following statements.}

\vspace{4mm}

\begin{itemize}

\item[(a)] $$\pFq{2}{1}{\frac{5}{2}&\frac{3}{2}}{&\frac{1}{2}}{1} = \frac{\Gamma(\frac{1}{2})\Gamma(\frac{1}{2})}{\Gamma(-2)\Gamma(-1)} = 2 \pi.$$

\vspace{4mm}

\item[(b)] $$\pFq{2}{1}{\frac{1}{2}&\frac{3}{2}}{&\frac{7}{2}}{1} = \frac{15 \pi}{32}.$$

\end{itemize}

\vspace{4mm}

\hrule{}

\vspace{8mm}

\item[(3)] Let $B_{k}$ denote the $k$-th Bernoulli number. Show that $B_{2k+1} = 0$ for $k \geq 1$.

\vspace{4mm}

\hrule{}

\vspace{8mm}

\item[(4)] Let $s \in \C$. Recall the Riemann zeta function is defined as 

$$\zeta(s) = \sum_{n=1}^{\infty} \frac{1}{n^s}$$

when $\text{Re}(s) > 1$.

\vspace{4mm}

\begin{itemize}
    \item[(a)] Let $p$ be a prime. Show that 

    $$\zeta(s) = \prod_{p} \frac{1}{1 - \frac{1}{p^s}} $$

    when $\text{Re}(s) > 1$, where $\displaystyle \prod_{p}$ denotes the product over all primes $p$.

\vspace{4mm}

   \item[(b)] Use the functional equation for $\zeta(s)$,

   $$\zeta(s) = 2^{s}\pi^{s-1} \sin \bigg(\frac{\pi s}{2}\bigg) \Gamma(1-s)\zeta(1-s)$$

   to show that $\zeta(-2k) = 0$ for all positive integers $k$. Why is $\zeta(2k)$ nonzero for all positive integers $k$?

\vspace{4mm}

   \item[(c)] A known formula for $\zeta(s)$ at positive even integers is

   $$\zeta(2k) = \frac{(-1)^{k+1}B_{2k}2^{2k-1}\pi^{2k}}{(2k)!}.$$

   \vspace{4mm}

   Use the functional equation for $\zeta(s)$ and the $\zeta(2k)$ formula to show that

   $$\zeta(1-2k) = (-1)^{2k-1} \frac{B_{2k}}{2k}$$

   for all positive integers $k$.
\end{itemize}

\end{itemize}

\vspace{8mm}

\hrule{}

\vspace{8mm}

\begin{center}
\large\textbf{\underline{Week II:}}
\end{center}

\vspace{8mm}

\section*{Lecture 4 Problems}

\begin{itemize}
    \item[\textbf{1.}] Show that $\fp^{\times}$ and $\fphat$ are isomorphic as groups.

    \vspace{8mm}

    \hrule{}

    \vspace{4mm}

    \item[\textbf{2.}] Prove both versions of the orthogonality of characters on $\fp^{\times}$ or look up the proofs. 

    \vspace{8mm}

    \hrule{}

    \vspace{4mm}


    \item[\textbf{3.}] Let $A,B \in \fphat$ and recall the Gauss and Jacobi sums,

    $$g(A) = \sum_{x \in \fp^{\times}} A(x) \zeta_{p}^{x}$$

    and

    $$J(A,B) = \sum_{x \in \fp^{\times}} A(x)B(1-x),$$

    respectively.

    \vspace{4mm}

    \begin{itemize}
    
    \item[(a)] Show that 

    $$J(A,B) = \frac{g(A)g(B)}{g(AB)}$$

    if $AB \neq \eps$.
    \vspace{4mm}

    \item[(b)] Salvage the result in part $(a)$ when $AB=\eps$.
    \end{itemize}

\vspace{8mm}

\hrule{}

\vspace{4mm}

\item[\textbf{4.}]

\begin{itemize}
\item[(a)] Show that $J(A,\overline{A}) = -A(-1)$ if $A \neq \eps$, where $\overline{A}$ denotes the conjugate of the character $A$.

\vspace{4mm}

\item[(b)] Show that $J(A, \eps) = -1$ if $A \neq \eps$.

\vspace{4mm}

\item[(c)] Salvage the results in part $(a)$ and $(b)$ when $A = \eps$. 
\end{itemize}

\vspace{8mm}

\hrule{}

\vspace{4mm}

\item[\textbf{5.}]

The \textit{double-angle formula} for Gauss sums says that 

$$g(A)g(\phi A) = g(A^{2})g(\phi)\overline{A}(4)$$

\vspace{2mm}

for every multiplicative character $A \in \fphat$.

\vspace{4mm}


\begin{itemize}
\item[(a)] Use the double-angle formula to show 

$$J(A,A) = \overline{A}(4)J(A, \phi).$$

\vspace{4mm}

\item[(b)] Use part $(a)$ to determine $J(\eps, \eps)$ and $J(\eps,\phi)$ in another way.

\vspace{4mm}

\item[(c)] Suppose $p \equiv 1 \mod 4$. Substitute $A = \eta_{4}\chi$ into the double angle formula and solve for $g(\phi \chi^{2})$.
\end{itemize}

\vspace{4mm}

\textbf{Remark:} The substitution in part $(c)$ is a commonly used tool for simplifying Gauss sums inside the $H_{p}$ function.

\vspace{8mm}

\hrule{}

\vspace{8mm}

\section*{Lecture 5 Problems}

\begin{itemize}
\item[\textbf{1.}] Compute 
\vspace{5mm}

\begin{itemize}

\item[(a)] $$H_{p}\left[\begin{matrix} \frac{1}{3} & \frac{2}{3} \smallskip \\   & 1 \end{matrix} \; ; \; 1 \right]$$ for primes $p \equiv 1 \mod 3$.
    \vspace{5mm}

    \vspace{5mm}

\item[(b)] $$H_{p}\left[\begin{matrix} \frac{1}{4} & \frac{3}{4} \smallskip \\   & 1 \end{matrix} \; ; \; 1 \right]$$ for primes $p \equiv 1 \mod 4$.
\end{itemize}

\vspace{8mm}

\hrule{}

\vspace{4mm}

\item[\textbf{2.}] Let $\alpha = \{a_{1}, \ldots, a_{n}\}$ and $\beta = \{1,b_{2}, \ldots, b_{n}\}$ be the first and second rows of the hypergeometric data, respectively, with all $a_{i}, b_{i} \in \Q$.
\\

The hypergeometric data $\{\alpha, \beta\}$ is \underline{primitive} if $a_{i}-b_{j} \notin \Z$ for all $i,j \in [1,n]$. 

\vspace{4mm}

Another property of hypergeometric data is to be defined over $\Q$. We say a multiset, say $\alpha$, is defined over $\Q$ if 

$$\prod_{j=1}^{n} (X - e^{2 \pi ia_{j}}) \in \Z[x],$$

where the $a_{j} \in \alpha$.

\vspace{4mm}

The hypergeometric data $\{\alpha, \beta\}$ is \underline{defined over $\Q$} if both multisets, $\alpha$ and $\beta$, are defined over $\Q$.

\vspace{4mm}

In each case determine if the hypergeometric data is algebraic, primitive, and defined over $\Q$.

\vspace{4mm}

\begin{itemize}
\item[(a)] $\alpha = \{\frac{1}{2}, \frac{1}{4}, \frac{3}{4}, \frac{1}{6} \}$ and $\beta = \{1, \frac{3}{2}, \frac{1}{5}, \frac{4}{5}\}$.

\vspace{4mm}

\item[(b)] $\alpha = \{ \frac{1}{2}, \frac{1}{6}, \frac{5}{6}\}$ and $\beta = \{1, \frac{1}{3}, \frac{2}{3}\}$.

\vspace{4mm}

\item[(c)] $\alpha = \{\frac{1}{8}, \frac{3}{8}, \frac{5}{8}, \frac{7}{8}\}$ and $\beta = \{1, \frac{1}{2}, \frac{1}{4}, \frac{3}{4}\}$.

\vspace{4mm}

\item[(d)] $\alpha = \{\frac{1}{2}, \frac{1}{3}, \frac{2}{3}\}$ and $\beta = \{1, 1, 1\}$.

\end{itemize}

\vspace{8mm}

\hrule{}

\vspace{4mm}

\item[\textbf{3.}] Recall the algebraic formula from the lecture,

$$\pFq{2}{1}{a & a+\frac{1}{2}}{&\frac{1}{2}}{z} = \frac{1}{2} \bigg[(1 + \sqrt{z})^{-2a} + (1 - \sqrt{z})^{-2a} \bigg]$$

\vspace{4mm}

where $0 < a < \frac{1}{2}$ and $z \in \C$ with $|z| < 1$.

\vspace{4mm}

\begin{itemize}
\item[(a)] Let $a = \frac{1}{3}$ in the algebraic formula above and think about what the finite field analog of this formula should be.

\vspace{4mm}

\item[(b)] Show that if $p \equiv 1 \mod 6$ and $\l \in \fp^{\times}$ then

$$H_{p}\left[\begin{matrix} \frac{1}{3} & \frac{5}{6} \smallskip \\   & \frac{1}{2} \end{matrix} \; ; \; \l \right] = \bigg(\frac{1 + \phi(\l)}{2} \bigg) \bigg[\eta_{3}(1 + \sqrt{\l}) + \eta_{3}(1 - \sqrt{\l}) \bigg].$$
\end{itemize}

\vspace{8mm}

\hrule{}

\vspace{4mm}

\item[\textbf{4.}] In each part simplify your answer to the best of your ability.

\vspace{4mm}

\begin{itemize}
\item[(a)] Compute 

$$H_{p}\left[\begin{matrix} \frac{1}{3} & \frac{2}{3} \smallskip \\   & \frac{1}{2} \end{matrix} \; ; \; 2 \right]$$ for the primes $p = 7, 13,$ and $19$.

\vspace{4mm}

\item[(b)] Compute 

$$H_{p}\left[\begin{matrix} \frac{1}{6} & \frac{5}{6} \smallskip \\   & \frac{1}{2} \end{matrix} \; ; \; 4 \right]$$ for the primes $p = 13, 17,$ and $61$.

\vspace{4mm}

\item[(c)] Compute 

$$H_{5}\left[\begin{matrix} \frac{1}{4} & \frac{3}{4} \smallskip \\   & \frac{1}{2} \end{matrix} \; ; \; \l \right]$$ for all $\l \in \mathbb{F}_{5}^{\times}$.
\end{itemize}

\vspace{8mm}

\hrule{}

\vspace{8mm}

\section*{Lecture 6 Problems}

\begin{itemize}
\item[\textbf{1.}] Let $\tilde{E_{\l}}$ denote the mod $p$ reduction of the Legendre elliptic curve 
$$E_{\l}: y^{2} = x(1-x)(1-\l x)$$ where $\l \in \Z \setminus \{0,1\}$ and $p$ is a prime of good reduction. Recall that $|\tilde{E_{\l}}(\fp)|$ is the number of solutions to $\tilde{E_{\l}}$ over $\fp$ plus one, for the point at infinity.

\vspace{4mm}

\begin{itemize}
\item[(a)] Compute $\tilde{E_{\l}}(\mathbb{F}_{11})$ for all $\l \in \F_{11} \setminus \{0,1\}$. Do you see any patterns? Any conjectures?

\vspace{4mm}

\item[(b)] Use your computations in part $(a)$ to determine the values of 

$$H_{11}\left[\begin{matrix} \frac{1}{2} & \frac{1}{2} \smallskip \\   & 1 \end{matrix} \; ; \; \l \right]$$

for all $\l \in \fp \setminus \{0,1\}$. Any patterns or conjectures for these $H_{11}$ values?

\vspace{4mm}

\item[(c)] Repeat parts $(a)$ and $(b)$ with $p=17$ and use your computations to refine your conjectures from parts $(a)$ and $(b)$.
\end{itemize}

\vspace{8mm}

\hrule{}

\vspace{4mm}

\item[\textbf{2.}] Read \underline{\href{https://www.ams.org/journals/bull/2008-45-02/S0273-0979-08-01207-X/S0273-0979-08-01207-X.pdf}{Mazur's article}} to learn more about the Sato-Tate conjecture and related equidistribution problems.

\vspace{8mm}

\hrule{}

\vspace{4mm}

\item[\textbf{3.}] Read \underline{\href{https://www.ams.org/journals/bull/2008-45-02/S0273-0979-08-01207-X/S0273-0979-08-01207-X.pdf}{this summary}} on the introduction of hypergeometric moments by Ono, Saad, and Saikia.

\vspace{8mm}

\hrule{}

\end{itemize}












    
\end{itemize}

\end{document}
